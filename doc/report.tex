\documentclass[a4paper]{article}
\usepackage[T1]{fontenc}
\usepackage[utf8]{inputenc}
\usepackage[top=2cm, bottom=2cm, left=2cm, right=2cm]{geometry}
\usepackage{color}
\usepackage{url}
\usepackage{array}
\usepackage{pifont}
\usepackage{graphicx}
\setlength{\skip\footins}{2cm}

\title{COMP-512 Distributed Systems, Fall 2011}
\author{
Matthieu \textsc{Dubet} \\
Alexander \textsc{Kawrykow} \\ \\ \\
\emph{School of Computer Science, McGill University }
}

\begin{document}
\maketitle
%\includegraphics{mcgill_logo.png}

\clearpage
\tableofcontents
\clearpage
\section{Introduction}
\section{Project Part 1: Distributing an Application}
\subsection{Design}



\subsection{Testing}
\subsubsection{Bash scripts}
To facilitate the testing of the system, Bash scripts are available.
\begin{itemize}
\item{
{\tt launch.sh [tcp/rmi] [car/room/flight/middleware] [port] [.....]}.

It take several arguments to launch each of the service (CarResourceManager,Room,Flight,Middleware or Client), either via TCP or RMI.
}
\item{
{\tt launch-localhost.sh [tcp/rmi] [portcar] [portroom] [portflight] [portmiddleware]}.

It deploys all the services all at a time on localhost.
}
\item{
{\tt launch-client.sh [tcp/rmi] [host:port]} is used to launch a client (either RMI or TCP) 
allowing fast testing when it's combined with the {\tt input1} file which contains multiple testcases.
}
\end{itemize}

\subsubsection{Example}
A traditional runcase with 3 servers would be :

First, login to server1 and launch all the resource managers.

{\tt ./launch.sh rmi flight 2000}

{\tt ./launch.sh rmi car 2001}

{\tt ./launch.sh rmi room 2002}

Then, login to server2 and launch the middleware.

{\tt ./launch.sh rmi middleware 2003 -flight=server1:2000 -car=server1:2001 -room=server1:2002}

Finaly, login to server3 and use the client redirecting stdin from the input1 file
 
{\tt ./launch-client.sh rmi client server2:2003 < input1}



%\bibliographystyle{plain}
%\bibliography{bib}
\end{document}
